\section{Diskussion}
\label{sec:Diskussion}
Bie diesem Versuch wird zuerst die Schallgeschwindigkeit in Acryl mittels des Impuls-Echo-Verfahrens und Durchschallungs-Verfahren bestimmt.
Daraus wird keine Aussage über die Richtigkeit der Schallgeschwindigkeit beim Vergleich von zwei Verfahren getroffen.
 Der Grund ist beim Durchschallungs-Verfahren werden die Messdaten aus der anderen Quelle \cite{1} genommen (Wegen des Defektes einer Sonde).
\begin{align*}
    \text{Impuls-Echo-Verfahren}&:\,\, c= (2406,137\pm 449,721)\text{m/s}\\
    \text{Durchschallungs-Verfahren}&:\,\,c= (1624,243\pm 154,277)\text{m/s}\\
    \text{Literatur}&:\,\, c =2730 \, \text{m/s}.\text{\cite{s}}\\
   \end{align*}
Somit ergeben sich nach  
\begin{equation}
    \text{Abweichung} = \Big\vert\frac{{\text{Berechnete Werte}}-\text{Literaturwerte}}{\text{Literaturwerte}}\Big\vert \cdot 100 \%
\end{equation}
die Abweichungen von jeweils 
\begin{align*}
    \text{Impuls-Echo-Verfahren}&:\,\,11,863  \,\% \\
    \text{Durchschallungs-Verfahren}&:\,\,40,504 \,\% .\\  
\end{align*} 
Der materialspezifische Koeffizient von Acryl wird durch die Untersuchung der Dämpfung mit dem Impuls-Echo-Verfahren bestimmt.
\begin{align*}
\alpha &=(13,062\pm2,068)\, \text{m}^{-1}.\\
\end{align*} 
Die Dicken zweier Polyacrylplatten werden mit dem Impuls-Echo-Verfahren, aus der FFT-Funktion und aus dem Cepstrum bestimmt und  mit den Werten, die mit der Schieblehre ausgemessen werden, verglichen.
Das Cepstrum-Verfahren wird daher auch zur Bestimmung von Längen verwendet.
    \begin{align*}
        d_{\text{dünn}}&= 9,943\, \text{mm}  &   d_{\text{dünn},\,\text{theorie}} &= 9,9\, \text{mm} \\
        d_{\text{dick}}&= 11,850\, \text{mm}  &   d_{\text{dick},\,\text{theorie}} &= 12,8 \, \text{mm} \\
    \end{align*}
    Es ergeben sich die Abweichungen 
  \begin{align*}
    \text{für dicke Platte}&: 0,43\, \%\\
    \text{für dünne Platte}&: 7,42\, \%.\\
    \end{align*}

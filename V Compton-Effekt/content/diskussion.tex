\section{Diskussion}
\label{sec:Diskussion}
\paragraph{}
Bei der Aufnahme eines Emissionspektrums der Kupfer Röntgenröhre werden Intensitäten und Kristallwinkeln bei dem ersten und zweiten Bergen (entsprechen für $K_\alpha$- und $K_\beta$-Linie von Kupfer Emissionspektrum) erkannt.
Die entsprechende Energien werden  mit Literaturwerten zu verglichen, wurden diese aus \cite{XT} entnommen\\
\begin{align*}
    E_{\alpha} &= 8,059 \,\mathrm{keV}  &   E_{\alpha,\,\text{Literatur}} &= 8,048 \,\mathrm{keV} \\
    E_{\beta} &= 8,931\,\mathrm{keV}  & E_{\beta,\,\text{Literatur}} &= 8,905 \,\mathrm{keV} .\\
\end{align*}


Somit ergeben sich nach  
\begin{equation}
    \text{Abweichung} = \Big\vert\frac{E_{\alpha, \text{berechnet}}-E_{\alpha, \text{Literatur}}}{E_{\alpha,\text{ Literatur}}}\Big\vert \cdot 100 \%
\end{equation}
die Abweichungen von jeweils 
\begin{align*}
    E_{\alpha} &: 0,136 \,\% \\
    E_{\beta} &: 0,291 \,\% .\\
\end{align*} 

Die Compton-Wellenlänge der Kupfer Röntgenröhre wird über die Vergrößerung der Wellenlänge der ungestreuten Röntgenstrahlung und der gestreuten Röntgenstrahlung bestimmt und wird mit dem theoretischen Wert zu vergleichen.
\begin{align*}
    \lambda_c &= 3,751 \,\mathrm{pm}\\
    \lambda_{\text{theorie}} &=\frac{h}{m_e \cdot c} = 2,426 \cdot 10^{-12}\,\mathrm{pm}\\
    \mathrm{Abweichung}&: 35,324 \, \% \\
\end{align*} 

Bei der Rechnungen wurden die folgende Einfüsse (systematische Fehler) nicht berücksichtigt:
\begin{enumerate}
    \item Ein Fehler durch die Messung des Stroms über den Messverstäker kann ebenfalls als unwahrscheinlich angenommen werden, da vor jeder Einzeilmessung der Verstärker auf null justiert wurde.
    Die Justage führt allerdings zu kleinen, aber statistischen Abweichungen.
    \item Fehler in der optischen Anordnung sind zwar nicht ausgeschlossen, tragen aber auch nicht zu einer Abweichung der Ausgleichungsgeradensteigung bei, scheiden also als Grund für den systematischen Fehler aus.
\end{enumerate}

Bei geringen Zählraten ist überdies die Totzeit $\tau$ des Zählrohres nicht nötig zu berücksichtigen, da von diesem alle einfallenden Photonen registriert werden.

Der Compton-Effekt kann in sichtbaren Bereich des Spektrums nicht auftreten. Der Grund davon ist:
Die Wellenlängenänderung (Compton-Wellenlänge) ist konstant und liegt in Picometerbereich.
Die Größenordnung der Wellenlänge dieses Bereichs liegt zwischen 780nm und 380nm (Namometerbereich).
Die Wellenlängenänderung ist daher zum Vergleich mit sichtbaren Wellenlänge sehr klein, nicht bemerkbar.


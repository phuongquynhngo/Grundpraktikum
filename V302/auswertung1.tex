\section{Auswertung}
\label{sec:Auswertung}
\subsection{Statische Methode}
In den folgenden Abbildungen sind die Temperaturkurven der Thermoelemente T1, T4, T5 und T8 dargestellt.
Daraus können Schlüsse über die Wärmeleitung der einzelnen Stäbe gezogen werden.
Alle Temperaturkurven steigen monoton und sind nach oben beschränkt.
Im Vergleich der fernen Thermoelemente T1 und T4 in Abbildung[2] des breiten und des schmalen Messingstabes fällt auf,
dass die Kurve des schmalen Stabes weniger stark steigt, als die des breiten Messingstabes. Daraus kann eine bessere Wärmeleitung des breiten Stabes 
abgeleitet werden. Dieses Ergebnis liefert auch die Gleichung[1].
In Abbildung[3] werden die Temperaturkurven von Aluminium T5 und Edelstahl T8 verglichen. Die Kurve des Aluminiumstabes zeigt eine deutlich größere Steigung 
als die Edelstahlkurve. Aluminium leitet also besser Wärme als Edelstahl.
\begin{figure}[H]
  \centering
  \includegraphics[width=\textwidth]{700secstat14Messing.pdf}
  \caption{Temperaturverläufe der Thermoelemente T1(Messing,breit) und T4(Messing,schmal) in einem Zeitraum von 700 s.}
  \label{fig:2}
\end{figure}
\begin{figure}[H]
  \centering
  \includegraphics[width=\textwidth]{700secstat58AluStahl.pdf}
  \caption{Temperaturverläufe der Thermoelemente T5(Aluminium) und T8(Edelstahl) in einem Zeitraum von 700 s.}
  \label{fig:3}
\end{figure}

Anhand der Temperaturen der fernen Thermoelemente nach 700 s in Tabelle[1] kann die Wärmeleitung der Stäbe verglichen werden.
Der Aluminiumstab hat nach 700 s die größte Temperatur, daraus wird abgeleitet, dass Aluminium die beste Wärmeleitung hat. Die schlechteste
Wärmeleitung hat Edelstahl, da der Edelstahlstab nach 700 s die geringste Temperatur hat.
Außerdem zeigen die Werte eine bessere Wärmeleitung des breiten Messingstabes als des schmalen Messingstabes, welches durch Gleichung [1] bestätigt wird.
Die Temperaturen nach 700 s bestätigen die Ergebnisse aus der Auswertung der Abbildungen[2] und [3].
\begin{table}[H]
  \centering
  \caption{Temperaturen der fernen Thermoelemente nach 700s}
  \label{tab:1}
  \begin{tabular}{c c c c}
    \toprule
    $T_{Messing,breit}$ in $°C$ &   $T_{Messing,schmal}$ in $°C$ &  $T_{Aluminium}$ in $°C$ &  $T_{Edelstahl}$ in $°C$ \\
    \midrule
    48,14 & 41,11 & 52,45 & 37,04\\
    \bottomrule
  \end{tabular}
\end{table}  
\begin{figure}[H]
  \centering
  \includegraphics[width=\textwidth]{700secstatdifferenz.pdf}
  \caption{Temperaturdifferenz zwischen dem Ende T8 und dem Anfang T7 des Edelstahlstabs und Temperaturdifferenz zwischen dem Ende T1 und dem Anfang T2 des breiten Messingstabs.}
  \label{fig:4}
\end{figure}
Aus Abbildung [4] lässt sich entnehmen, dass der breite Messingstab eine bessere Wärmeleitung als der Edelstahlstab besitzt, da die Temperaturdifferenz
zwischen den Thermoelementen T7 und T8 deutlich größer ist. \\
Mit Gleichung[1] und den Werten aus den folgenden Tabellen werden die Wärmeströme $\Delta Q / \Delta t$ berechnet.
Die Entfernung $\Delta x = 0,03 \text{m}$ wird gemessen.
\begin{table}[H]
  \centering
  \caption{Werte zur Berechnung der Wärmeströme und Wärmeleitfähigkeit}
  \label{tab:2}
  \begin{tabular}{c c c c c}
  \toprule
  Material & $\kappa $ in $\frac{W}{mK}$ & $\rho$ in $\frac{kg}{m^3}$ & $c$ in $\frac{J}{kg\cdot K}$ & $A$ in $m^2$\\
  \midrule
    Messing & 120 & 8520 & 385 & 4,8 $\cdot 10^{-5}$ \\
    Messing & 120 & 8520 & 385 & 2,8 $\cdot 10^{-5}$ \\
    Aluminium & 236 & 2800 & 830 & 4,8 $\cdot 10^{-5}$ \\
    Edelstahl & 53 & 8000 & 400 & 4,8 $\cdot 10^{-5}$ \\
  \bottomrule
\end{tabular}
\end{table}
\begin{table}[H]
  \tiny
  \centering
  \caption{Wärmeströme}
  \label{tab:3}
  \begin{tabular}{c c c c c c c c c}
  \toprule
  Zeit in s & $T_2 - T_1$ in K & $T_3 - T_4$ in K  & $T_6 - T_5$ in K  & $T_7 - T_8$ in K  & $\frac{dQ_{M,breit}}{dt}$ in W & $\frac{dQ_{M,schmal}}{dt}$ in W & $\frac{dQ_{A}}{dt}$ in W & $\frac{dQ_{S}}{dt}$ in W\\
  \midrule
    100 & 5,2 & 3,02 &2,07 & 7,28&-0,0998&-0,338&-0,782&-0,617\\
    200 & 4,1 & 2,9 &1,01&9,95&-0,787&-0,325&-0,381&-0,844\\
    300 & 3,46 & 2,72 &0,57&10,16&-0,664 &-0,305&-0,215&-0,862\\
    500 & 3,02 & 2,57 &0,3&9,91&-0,580&-0,288&-0,113&-0,840\\
    700 & 2,96 & 2,59 &0,25&9,88&-0,568&-0,290 &-0,094 & -0,838\\
  \bottomrule
\end{tabular}
\end{table}
\subsection{Dynamische Methode}
Bei der dynamischen Methode wird die Wärmeleitfähigkeit aus der Ausbreitungsgeschwindigkeit der Temperaturwelle bestimmt.
Die Stäbe werden in einer Periode $T_{P1} = 80 s $ geheizt und abgekühlt.
Es wird zehn Perioden lang gemessen und die in Tabelle [4] enthaltenen Werte von Abbildung [5] abgelesen.
Mit diesen Werten wird die Wärmeleitfähigkeit $\kappa_{Messing}$ berechnet.
Die Wärmeleitfähigkeit $\kappa_{Aluminium}$ wird ebenfalls berechnet mit den Werten aus Tabelle[5], die in Abbildung [6] abgelesen werden.
Für Edelstahl werden die gleichen Größen gemessen und berechnet, jedoch wird der Stab dabei in einer Periode von $T_{P2} = 200 s$ geheizt und gekühlt.
\begin{figure}[H]
  \centering
  \includegraphics[width=\textwidth]{80secdyn12Messing.pdf}
  \caption{Temperaturwellen Messing, 80 s Periodendauer}
  \label{fig:5}
\end{figure}
\begin{table}[H]
  \centering
  \caption{Berechnung von $\kappa_{Messing}$}
  \label{tab:4}
  \begin{tabular}{c c c c c}
    \toprule
   Periode &   $A_{nah,T1}$ in K &  $A_{fern,T2}$ in K &  $\Delta t$ in s & $\kappa$ in $\frac{W}{mK}$ \\
    \midrule
    1 & 4 & 0,6 & 8 & 97,26 \\
    2 & 4,5 & 0,8 & 10 & 85,46 \\
    3 & 4,8 & 1,5 & 10 & 126,90 \\
    4 & 4,8 & 1,1 & 11 & 91,08 \\
    5 & 5 & 1,2 & 11 & 94,03 \\
    6 & 5,1 & 1,2 & 12 & 85,01 \\
    7 & 5 & 1,4 & 12 & 96,63 \\
    8 & 4,9 & 1,2 & 12 & 87,43 \\
    9 & 4,9 & 1,2 & 13 & 80,71 \\
    \bottomrule
  \end{tabular}
\end{table}
Die Werte für $\kappa $ werden mit Python gemittelt und die Standardabweichung berechnet. $$\kappa_{Messing} =(93.834 \pm 12.831) \frac{W}{mK}$$
\begin{figure}[H]
  \centering
  \includegraphics[width=\textwidth]{80secdyn56Alu.pdf}
  \caption{Temperaturwellen Aluminium, 80 s Periodendauer}
  \label{fig:6}
\end{figure}
\begin{table}[H]
  \centering
  \caption{Berechnung von $\kappa_{Aluminium}$}
  \label{tab:5}
  \begin{tabular}{c c c c c}
    \toprule
   Periode &   $A_{nah,T6}$ in K &  $A_{fern,T5}$ in K &  $\Delta t$ in s & $\kappa$ in $\frac{W}{mK}$ \\
    \midrule
    1 & 3,6 & 1,3 & 7 & 146,68 \\
    2 & 4,2 & 2 & 7 & 201,36 \\
    3 & 4,7 & 2,3 & 8 & 189,92\\
    4 & 4,8 & 2,6 & 10 & 170,57\\
    5 & 5,1 & 2,5 & 9 & 162,98 \\
    6 & 5 & 2,5 & 9 & 167,64 \\
    7 & 4,9 & 2,9 & 10 & 199,38 \\
    8 & 5 & 2,7 & 9 & 188,58 \\
    9 & 5,1 & 2,9 & 9 & 205,84 \\
    \bottomrule
  \end{tabular}
\end{table}
$$\kappa_{Aluminium} =(181.439 \pm 19.133) \frac{W}{mK}$$
\begin{figure}[H]
  \centering
  \includegraphics[width=\textwidth]{200secdyn78Edelstahl.pdf}
  \caption{Temperaturwellen Edelstahl, 200 s Periodendauer}
  \label{fig:7}
\end{figure}
\begin{table}[H]
  \centering
  \caption{Berechnung von $\kappa_{Edelstahl}$}
  \label{tab:6}
  \begin{tabular}{c c c c c}
    \toprule
   Periode &   $A_{nah,T7}$ in K &  $A_{fern,T8}$ in K &  $\Delta t$ in s & $\kappa$ in $\frac{W}{mK}$ \\
    \midrule
    1 & 9,5 & 0,5 & 45 & 10,87 \\
    2 & 10,5 & 0,5 & 45 & 10,51 \\
    3 & 11 & 1 & 45 & 13,35\\
    4 & 12 & 1 & 48 & 12,07\\
    5 & 13 & 1,5 & 48 & 13,89 \\
    6 & 13 & 1,5 & 48 & 13,89 \\
    7 & 13 & 2 & 50 & 15,39 \\
    8 & 13 & 2 & 50 & 15,39 \\
    9 & 13 & 2 & 50 & 15,39 \\
    \bottomrule
  \end{tabular}
\end{table}
$$\kappa_{Edelstahl} =(13.417 \pm 1.791) \frac{W}{mK}$$
Mit den berechneten Werten für $\kappa$ können die Wellenlängen und die Frequenzen der Temperaturwellen über folgende Formeln bestimmt werden.
\begin{align*}
\lambda &= \sqrt{ \frac{4 \pi \kappa T}{\rho c}}\\
f &= \frac{1}{T}
\end{align*}
Da $\kappa$ durch gewisse Messunsicherheiten eine fehlerbehaftete Größe ist, muss zur Wellenlänge $\lambda$ ein Fehler $\Delta \lambda$ mithilfe der Gaußschen Fehlerfortpflanzung berechnet werden.
$$\Delta \lambda = \sqrt{\Bigl(\frac{2 \pi T}{\rho c} \cdot \Bigl(\frac{4 \pi \kappa T}{\rho c}\Bigr)^{-\frac{1}{2}}\Bigr)^2 \cdot \Delta \kappa^2} $$
Es werden folgende Wellenlängen und Frequenzen berechnet.
\begin{align*}
  \lambda_{Messing} &= (0,170 \pm 0,012 )\ \text{m}\\
  f_{Messing} &= 0,0125 \ \text{Hz}\\
  \lambda_{Aluminium} &= (0,280 \pm 0,015 ) \ \text{m}\\
  f_{Aluminium} &= 0,0125 \ \text{Hz}\\
  \lambda_{Edelstahl} &= (0,103 \pm 0,0069 ) \ \text{m}\\
  f_{Edelstahl} &= 0,005 \  \text{Hz}
\end{align*}




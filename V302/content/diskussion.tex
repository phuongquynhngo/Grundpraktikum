\section{Diskussion}
Wenn man die statische Fehler mit den Gaußfehler vergleicht konnte festgestellt werden, dass der Gaußfehler größer
als die statische Fehler sind, daher sind die Gaußfehler die relevanten Fehler.
In Allgemeinen kann gesagt werden, dass es sich um kleine Fehler handelt aufgrund der Fehler der 
Bauteile. Um genauere Ergebnisse zu erhalten, müssen mehrere Verusche durchgeführt werden.
Zu den Wien-Robinson-Brücke kann gesagt werden, dass die Messwerte auf der Theoriekurve liegen, wie es in der Abbildung zu sehen ist.
Daher kann gesagt werden, dass die Messergbnisse gut sind.
Der Klirrfaktor liegt bei $$0,1154$$, welches ein schlechter Wert ist.



\label{sec:Diskussion}

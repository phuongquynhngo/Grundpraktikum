\section{Fehlerrechnung}
Der Mittwelwert und die Standardabweichung werden mit Python berechnet.
Nach Herstellerangaben beläuft sich der unsystematische relative Fehler aller 
Referenzbauteile aus \pm 0,2\% und der Quotient $\frac{R_3}{R_4}$ zeigt eine
unsystematische Abweichung bis zur $r_{\frac{R_3}{R_4}} = \pm 0,5 \%$.
Daher lässt der Gauß-Fehler sich berechnen aus:
\begin{align}
    \sigma = x_i \sqrt{(\Delta x)^2 + (\Delta y)^2}\,.
\end{align}
\subsection{Wheatstonsche Brücke}
In \ref{ssec:wheat} wurde R_2 variiert. Nun kann nach (),(),() berechnet werden.
Die Werte von $R_11$ und $R_10$ werden mit mit Python gemittelt und die Standardabweichung berechnet:
\begin{align*}
    R_{11} = (488,51 \pm 0,5623)
    R_{10} = (420,46 \pm 2,2640)
\end{align*}
\subsection{Kapazitätsbrücke}
\begin{align*}
    C_1= (652,12 \pm 16,9408)
    C_3= (420,46 \pm 2,2640)
\end{align*}
\subsection{Induktivitätsmessbrücke}


\label{sec:Fehlerrechnung}
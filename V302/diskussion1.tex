\section{Diskussion}
\label{sec:Diskussion}
\subsection{Statische Methode}
Mithilfe der gewonnen Messwerte werden die Graphen für die entfernten Thermoelemente in den Abbildungen
[2] und [3] dargestellt. Aus Abbildung[2] wird eine bessere Wärmeleitung des breiten Messingstabes abgeleitet, dieses Ergebnis bestätigt Gleichung [1].
Aluminium hat eine bessere Wärmeleitung als Edelstahl wie in Abbildung [3] zu erkennen ist.
Zudem wurden die Temperaturdifferenzen der unterschiedlich weit entfernten Messstellen des Messing- und des Edelstahlstabes in Abbildung [4] verglichen, um die bereits genannten Schlüsse zu bestätigen.\\
Da aus den statischen Messwerten keine Literaturwerte berechnet werden können, bestätigen die Ergebnisse nur grob die theoretischen Werte.
\subsection{Dynamische Methode}
Aus den erhaltenen Messwerten werden die Temperaturwellen für Messing, Aluminium und Edelstahl in den Abbildungen [5],[6] und [7]
dargestellt. Mithilfe dieser wurden die Wärmeleitfähigkeitskoeffizienten $\kappa$ berechnet.
Die prozentualen Abweichungen der berechneten $\kappa$ von den Literaturwerten sind in Tabelle[7] zu finden.
\begin{table}[H]
    \centering
    \caption{Abweichungen von $\kappa$}
    \label{tab:7}
    \begin{tabular}{c c c c}
      \toprule
     Material & $\kappa_{experimentell} \ \text{in} \ \frac{W}{mK} $ & $ \kappa_{Literatur} \ \text{in} \ \frac{W}{mK} $ &  Prozentuale Abweichung \\
      \midrule
      Messing & 93,834 &  120 & 21,81 \\
      Aluminium & 181,439 & 236 & 23,12 \\
      Edelstahl & 13,417 & 53 & 74,68 \\
      \bottomrule
    \end{tabular}
\end{table}
Die Abweichungen lassen sich durch Messunsicherheiten erklären.
Große Fehlerquellen sind das Ablesen der Amplituden und Phasendifferenzen anhand der Graphen, das Betätigen des "heat-/cool" -Schalters nach der halben Periodendauer oder das generelle Isolieren.
Hinzu kommen erschwerte Versuchsbedingungen, da in unmittelbarer Nähe Experimente mit Heizplatten durchgeführt wurden.
Die große Abweichung der Wärmeleitfähigkeit für Edelstahl lässt sich mit zwei Faktoren begründen.
Die Messung wurde zuletzt durchgeführt als sich die Umgebung über einen längeren Zeitraum durch die nahestehenden Heizplatten bereits erwärmt hatte und die Messergebnisse so beeinflusst wurden.
Außerdem ist der in Tabelle [7] angegebene Literaturwert der Wert für Stahl, daraus lässt sich ableiten, dass die Wärmeleitfähigkeit von Edelstahl zu der von Stahl verschieden ist.

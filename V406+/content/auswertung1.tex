\section{Auswertung}
\label{sec:Auswertung}
Zuerst wird der (thermische) Dunkelstrom  $I_\text{d}$ bei abgedeckter Detektorblende gemessen. 
Dieser Wert soll von dem gemessen Werten von Intensität $I$ abgezogen, um ein korrekten Messwert zu erhalten.
 $I_\text{d}$ beträgt zwar sehr kleinem Wert, sodass es vernachlässigbar ist.\\
 Des weiteren wird die Länge $L$ zwischen Spalt und Photodiode ermittelt. 
 Es ergibt sich den Wert von \(L=67 \, \text{cm}\).\\
Die Werte sin in allen felgenden Messungen identisch.
\subsection{Messung am Einfach-Spalt}
Die Breite von Einzelspalt ist mit dem Wert $b = 0,1 \, \text{mm}$ angegeben.\\
Die durch die Vermessung der Beugungsfigur erhaltende Werte und die berechnete Werte sind in Tabelle \ref{tab:einzel} dargestellt.
Den benögtigten Beugungswinkel errechnet sich nach
\begin{equation}
 \varphi \approx \frac{\xi-\xi_0}{L}\,,\\
    \end{equation}
wobei $\xi$ die Position der Messstelle, $\xi_0$ die Position des Hauptmaximums und
 $L$ die Entfernung zwischen Laser und Spalt darstellt.\\ 
Durch die Messung wird die Lage das Hauptmaximums zu $14\, \text{mm}$  bestimmt.
\begin{table}[H]
    \centering
    \caption{Die Messdaten am Einfachspalt}
    \label{tab:einzel}
    \begin{tabular}{| c | c |c||c|c|c| }
    \toprule
    $\xi/\mathrm{mm}$ &$I/\mathrm{\mu A}$  &  $\varphi$&$\xi/\mathrm{mm}$ &$I/\mathrm{\mu A}$  &  $\varphi$ \\
    \midrule
    0	    & 0,0150	& -0,0112   &  8,00   &    2,9500  & 0,0007\\   
    0,50	& 0,0095	& -0,0104   &  8,50   &    2,0000  & 0,0015\\ 
    1,00	& 0,0145	& -0,0097   &  9,00   &    0,9800  & 0,0022\\
    1,50	& 0,0285	& -0,0090   &  9,50   &    0,2980  & 0,0030\\
    2,00	& 0,0320	& -0,0082   &  10,00  &    0,1100  & 0,0037\\
    2,50	& 0,0200	& -0,0075   &  10,50  &    0,1750  & 0,0045\\
    3,00	& 0,0225	& -0,0067   &  11,00  &    0,2100  & 0,0052\\
    3,50	& 0,0670	& -0,0060   &  11,50  &    0,1400  & 0,0060\\
    4,00	& 0,1350	& -0,0052   &  12,00  &    0,0500  & 0,0067\\
    4,50	& 0,1200	& -0,0045   &  12,50  &    0,0280  & 0,0075\\
    5,00	& 0,0600	& -0,0037   &  13,00  &    0,0440  & 0,0082\\
    5,50	& 0,1400	& -0,0030   &  13,50  &    0,0440  & 0,0090\\
    6,00	& 0,6200	& -0,0022   &  14,00  &    0,0200  & 0,0097\\
    6,50	& 1,5000	& -0,0015   &  14,50  &    0,0100  & 0,0104\\
    7,00	& 2,7500	& -0,0007   &  15,00  &    0,0220  & 0,0112\\
    7,50	& 3,2000	& 0,0000    &  15,50  &    0,0360  & 0,0119\\

    \bottomrule
    \end{tabular}
  \end{table}
  \noindent
  \begin{table}
    \centering
    \caption{Messwerte des Einzelspalts.}
    \label{tab:einzel}
    \begin{tabular}{S[table-format=2.1] S[table-format=1.4]}
          \toprule
          {$\symup{Δx}\:/\:\si{\milli\meter}$}
          & {$\symup{I}\:/\:\si{\micro\ampere}$}\\
          \midrule
          19   & 0.013  \\
          19.5 & 0.0236 \\
          20   & 0.0385 \\
          20.5 & 0.042  \\
          21   & 0.037  \\
          21.5 & 0.042  \\
          22   & 0.1    \\
          22.5 & 0.245  \\
          23   & 0.445  \\
          23.1 & 0.51   \\
          23.2 & 0.55   \\
          23.3 & 0.59   \\
          23.4 & 0.62   \\
          23.5 & 0.66   \\
          23.6 & 0.70   \\
          23.7 & 0.73   \\
          23.8 & 0.75   \\
          23.9 & 0.77   \\
          24   & 0.78   \\
          24.1 & 0.78   \\
          24.2 & 0.76   \\
          24.3 & 0.76   \\
          24.4 & 0.74   \\
          24.5 & 0.72   \\
          24.6 & 0.7    \\
          24.7 & 0.66   \\
          24.8 & 0.63   \\
          24.9 & 0.58   \\
          25   & 0.54   \\
          25.5 & 0.3    \\
          26   & 0.135  \\
          26.5 & 0.042  \\
          27   & 0.025  \\
          27.5 & 0.036  \\
          28   & 0.042  \\
          28.5 & 0.031  \\
          29   & 0.0174 \\
          29.5 & 0.0098 \\
          30   & 0.013  \\
          \bottomrule
    \end{tabular}
\end{table}

\subsection{Messung am Doppeltspalt}
Analog zu der Einzelspaltmessung wird bem Doppeltspalt durchgeführt.
\begin{table}[H]
    \centering
    \caption{Die Messdaten am Einfachspalt}
    \label{tab:einzel}
    \begin{tabular}{| c | c |c||c|c|c| }
    \toprule
    $\xi/\mathrm{mm}$ &$I/\mathrm{\mu A}$  &  $\varphi$&$\xi/\mathrm{mm}$ &$I/\mathrm{\mu A}$  &  $\varphi$ \\
    \midrule
    0	    &0,0580	& -0,0090	& 7,00   & 5,4000 & 0,0015\\
    0,50	&0,0725	& -0,0082	& 7,50   & 4,5000 & 0,0022\\
    1,00	&0,0820	& -0,0075	& 8,00   & 1,6000 & 0,0030\\
    1,50	&0,0860	& -0,0067	& 8,50   & 0,2600 & 0,0037\\
    2,00	&0,0870	& -0,0060	& 9,00   & 0,2600 & 0,0045\\
    2,50	&0,1600	& -0,0052	& 9,50   & 0,4500 & 0,0052\\
    3,00	&0,2400	& -0,0045	& 10,00  & 0,3800 & 0,0060\\
    3,50	&0,2650	& -0,0037	& 10,50  & 0,2000 & 0,0067\\
    4,00	&0,2200	& -0,0030	& 11,00  & 0,1200 & 0,0075\\
    4,50	&0,3600	& -0,0022	& 11,50  & 0,1200 & 0,0082\\
    5,00	&3,6500	& -0,0015	& 12,00  & 0,1050 & 0,0090\\
    5,50	&3,8000	& -0,0007	& 12,50  & 0,0820 & 0,0097\\
    6,00    &7,0500 &  0,0000	& 13,00  & 0,0620 & 0,0104\\
    6,50	&6,4500	&  0,0007 & & \\			
 

    \bottomrule
    \end{tabular}
  \end{table}
  \noindent


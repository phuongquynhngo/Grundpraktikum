
\section{Diskussion}
\label{sec:Diskussion}
Die Abweichung wird im folgenden mit der Formel
\begin{equation}
	\nonumber
    \text{Abweichung} \,f = \Big\vert\frac{\text{Berechnete Werte}-\text{Theoretische Werte}}{\text{Theoretische Werte}}\Big\vert \cdot 100 \%
\end{equation}
berechnet.
Die bestimmten Messwerte und ihre Abweichungen von den bekannten Referenzwerten sind in Tabelle \ref{tab:Fehler} zu finden.\\
\begin{table}[H]
    \centering
    \caption{Die in der Auswertung bestimmten Messwerte mit den zugehörigen theoretischen Werten und Abweichungen}
    \label{tab:Fehler}
    \begin{tabular}{|c|c|c|c|c| }
		\toprule
		{Einzelspalt}&{Wert}&{gemessen}&{theoretisch }&{Abweichung} \\
		\midrule
		{}&$x_\text{0}$ & \SI{6.3\pm0.1e-4}\,\si{\meter} & - & - \\
		{}&$A$ & \SI{32,4\pm0,2}\,\si{\ampere\per\metre\squared} & - & - \\
		{}&$b$ & \SI{8.23\pm0.03e-5}\,\si{\meter} & \SI{7.5e-5}\,\si{\meter} & \SI{9.7}\,\si{\percent} \\
		\bottomrule
	\end{tabular}

  \end{table}

  \begin{table}[H]
    \centering
    \label{tab:Fehler1}
    \begin{tabular}{|c|c|c|c|c| }
		\toprule
		{Doppelspalt}&{Wert}&{gemessen}&{theoretisch }&{Abweichung} \\
		\midrule
		{}&$x_\text{0}$ & \SI{2.69\pm0.05}\,\si{\meter} & - & - \\
		{}&$A$ & \SI{8.12\pm0.07}\,\si{\ampere} & - & - \\
		{}&$b$ & \SI{1.57\pm0.01e-4}\,\si{\meter} & \SI{1.5e-4}\,\si{\meter} & \SI{4.7}\,\si{\percent} \\
		{}&$s$ & \SI{2.48\pm0.01e-4}\,\si{\meter} & \SI{2.5e-4}\,\si{\meter} & \SI{0.08}\,\si{\percent} \\
		\bottomrule
	\end{tabular}

  \end{table}
  \noindent
Die Interferenzmuster wurden so ausgerichtet, dass das Hauptmaximum jeweils bei $\Delta x =\SI{0}{\metre}$ liegt. Dass es dennoch zu Aweichungen der $x_0$ vom Nullpunkt kommt, lässt sich damit erklären, dass das Hauptmaximum nur per Augenmaß ausgerichtet wurde.\\
Die größeren Abweichungen der Spaltbreiten $b$  sind darauf zurückzuführen, dass bei dem Versuch folgende Einflüsse (systemmatische Fehler) nicht berücksichtigt werden:
\begin{enumerate}
	\item Abweichung bei der Lage des Hauptmaximums
    \item Ablesefehler bei Messgeräten
    \item Fehler in den Messgeräten bzw. Messmitteln
    \item Schwankungen der Netzspannung.
\end{enumerate}
Beim Vergleich der Interferenzmuster \ref{fig:Einzel} mit \ref{fig:Doppel} fällt zunächst auf, dass die Intensitäten des Doppelspalts um einen Faktor 10 größer sind als die des Einzelspalts.
 Dies ist auch sinnvoll, da dessen Spaltbreite $b_\text{Doppelspalt}=\SI{1,5e-4}{\metre}$ doppelt so groß ist wie die des Einzelspalts $b_\text{Einzelspalt}=\SI{7,5e-5}{\metre}$. 
 Außerdem lässt sich erkennen, dass das Muster des Doppelspalts durch den Intensitätsverlauf eines Einzelspalts eingehüllt werden kann.





\section{Durchführung}
\subsection{Aufbau}
Der Versuchsaufbau besteht aus einer Lichtquelle, einer Bank, auf der die verschiedenen
Linsen und der abzubildende Gegenstand gesetzt und verschoben werden kann, und einem
Schirm, auf dem das Bild sichtbar gemacht wird. Die Bank ist mit einer Skala versehen
und erlaubt dadurch eine Messung der Abstände zwischen den beteiligten Komponenten.
\subsection{Durchführung}
\subsection{Linse mit bekannter Brennweite}
Der Gegenstand und ein Schirm werden auf einer optischen Bank befestigt. Anfangs wird
die Gegenstandsweite $g$ festgesetzt. Damit ein scharfes Bild entsteht, wird der Schirm
verschoben. Dies wird für 14 Messungen durchgeführt und notiert.
\subsection{Bestimmung der Brennweite nach Bessel}
Anschließend wird nach Bessels Methode die Brennweite bestimmt. Der Aufbau bleibt
gleich. Dieses mal wird der Abstand $e$ zwischen Halogenlampe und Schirm festgesetzt.
Es werden $g_1$ und $b_1$ Werte notiert, welche ein scharfes Bild erzeugen. Anschließend wird
die Linse wieder verschoben bis weitere scharfe Bilder entstehen, dies sind die $g_2$ und $b_2$
Werte. Die Messung erfolgt für 11 Abstände $e$.
\subsection{Bestimmung der Brennweite nach Abbe}
Zuletzt wird nach Abbes Methode die Brennweite eines Linsensystems bestimmt. Hierfür
wird neben der Halogenlampe ein Gegenstand, eine Sammellinse ($f$ = 5 cm), eine
Zerstreuungslinse ($f$ = −10 cm) und ein Schrim auf der optischen Bank befestigt. Es
wird ein Referenzpunkt $A$ festgelegt indem beide Linsen aneinander gesetzt werden. Um
wieder ein scharfes Bild zu erzeugen, wird der Schirm verschoben. Die Abstände $g´$ und $b´$ werden notiert. Für die Berechnung des Abbildungsmaßstabes wird die Größe des
Gegenstandes vermessen. Durchgeführt wird die Messung für weitere 9 Gegenstandsweiten.

\label{sec:Durchführung}

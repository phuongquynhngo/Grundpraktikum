\section{Theorie}
\subsection{Widersprüche bei der klassischen Betrachtung des Photoeffektes}
Klassisch lässt der Photoeffekt durch die Annahme erklären, 
dass die Elektronen durch die einfallende elektromagnetische Welle zu Schwingungen angeregt werden. 
Dabei wird pro Zeiteinheit eine gewisse Energie auf das Elektron übertragen. 
Die Schwingungsamplitude (und somit die Energie) der Elektronen wird dadurch immer grö"ser,
 bis sie irgendwann ausreicht um sich aus dem Atom zu lösen.\\
Dementsprechend müssen die Energien der Elektronen abhängig von der Lichtintensität sein und
 auch bestimmte Resonanzfrequenzen müssen beobachtet bei denen der Photoeffekt besonders gut auftritt.\\
 Das Licht wird als elektromagnetische Welle betrachtet, egal bei welcher Wellenlänge $\lambda$ die Bestrahlung erfolgt,
  Elektronen austreten, solange nur lange genug abwartet bis die Welle (deren Energie homogen verteilt ist) eine Energie $E \geq A_k$ auf das Metall übertragen hat.\\
Die Beobachtungen stehen jedoch im Widerspruch zu diesen Annahmen
\begin{enumerate}
 \item Die Anzahl der ausgetretenen Elektronen (der Photostrom) ist abhängig von der Lichtintensität.
 \item Die Energie eines ausgetretenen Elektrons hängt nur von der Lichtfrequenz und nicht von der Intensität ab.
 \item Es existiert eine Grenzfrequenz, unterhalb der die Elektronen nicht austreten.
\end{enumerate}
\subsection{Die einsteinschen Korpuskulartheorie}
Eine Erklärung dieser Phänomene liefert dabei die Annahme, dass das monochromatische Licht auch als Teilchen (Photonen) betrachtet werden muss, 
die jeweils eine genau definierte Energiemenge
\begin{equation}
E_{\text{Ph}} = h \cdot \nu
\end{equation}
besitzen. 
Die Frequenz $\nu$, Wellenlänge $\lambda$ und Ausbreitungsgeschwindigkeit $c$ sind dabei über die Beziehung
\begin{equation}
c = \lambda \cdot \nu \Rightarrow \nu = \frac{c}{\lambda}
\end{equation}
mit einander verknüpft.\\
Das in Gleichung (1) auftretende $h$ ist dabei das Planck'sche Wirkungsquantum, 
der Proportionalitätsfaktor zwischen Energie und Frequenz des Photons.\\
Bei dieser Teilchentheorie des Lichtes kann das Photoeffekt als Sto"s eines Photons mit einem Elektron aufgefasst, 
das dadurch seine Energie an das Elementarteilchen übergibt. Wichtig ist dabei, das das Photon (zumindest beim Photoeffekt)
 immer nur seine gesamt Energie abgeben kann (inelastischer Sto"s) und es somit vernichtet wird.\\
Somit ist die Energiebilanz beim Phototoeffekt:
\begin{equation}
    h \nu  = {E}_{\text{kin}} + {A}_{k}.
\end{equation}
 Somit ist leicht ersichtlich wieso eine Grenzwellenlänge existiert ab der erst der Photoeffekt auftritt. 
 Denn ist die Energie geringer als die Austrittsarbeit (\(h\nu < A_k\)) reicht sie schlichtweg nicht aus um das Elektron aus dem Festkörper zu lösen.\\
Auch die Intensitätsabhängigkeit des Photostroms lässt sich somit erklären, dass bei höherer Intensitäten mehr Photonen 
auf die Oberfläche treffen und entsprechend mehr Elektronen ausgelöst werden.\\

\label{sec:Theorie}


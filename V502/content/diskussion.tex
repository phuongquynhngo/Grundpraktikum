\section{Diskussion}
Die magnetostatische Felder erzeugen Kräfte nur auf die relativ zum Feld bewegte Ladungen.
Geladene Teilchen (Elektronen, Protonen, Ionen) können sich in magnetischen Feldern bewegen und werden durch diese beeinflusst (Unter der Wirkung der Lorentzkräfte).
Dadurch kann die spezifische Ladung der Elektronen experimentell bestimmt werden.
Der Literaturwert der spezifische Ladung der Elektronen beträgt \(\frac{\text{e}_0}{\text{m}_0}=1,75882\,10^{11}\,\mathrm{\frac{C}{kg}}\) \cite{LT}.
Die Abweichung werden im folgenden mit der Formel
\begin{equation}
    \text{Abweichung $f$} = \Big\vert\frac{\text{Berechnete Werte} - \text{Literaturwerte}}{\text{Berechnete Werte}}\Big\vert \cdot 100 \%
\end{equation} berechnet\\
und es ergibt sich 
\begin{align*}
  \text{für}\,\, \frac{\text{e}_0}{\text{m}_0} &=(1,83685 \pm 0,05613)\, 10^{11}\,\mathrm{\frac{C}{kg}} : f=4,436\%\\
  \text{für}\,\, \frac{\text{e}_0}{\text{m}_0}& =(2,24675 \pm 0,16591)\,10^{11}\,\mathrm{\frac{C}{kg}}: f= 27,7419\% .\\
   \end{align*}
Eine Fehlerquelle stellt dar, dass die Messwerte sehr ungenau durch Untersuchung der Lage des Leuchtflecks abgelesen werden. 
Bei dem Verusch kann es nicht einen infinitesimal dünnen Elektronenstrahl sorgen, sodass es
genau genommen keinen exakten Radius gibt, sondern ein mittlerer Radius (Ablenkung) geschätzt werden muss.
Au"serdem wird dei dem Versuch Fehler in den Messgeräten bzw. Messmitteln nicht berücksichtigt.\\

\noindent In zweiten Teil des Versuch wird die Horizontalkomponente des Erdfeldes bestimmt.
 Der Wert davon lautet $\text{B}_{\text{hor}}=3,189 \,\mu \text{T}$.
 Die Totalintensität $\text{B}_{\text{total}}$ kann weiter durch die Bestimmung des  Inklinationswinkels $\varphi$ bestimmt werden.


\label{sec:Diskussion}

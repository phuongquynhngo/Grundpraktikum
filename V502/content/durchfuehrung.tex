\section{Durchführung}
\subsection{Bestimmung der spezifische Ladung der Elektronen}
Mit Hilfe einer gro"sen HELMHOLTZ-Spulen wird das homogenes Magnetfeld erzeugt.
Das Magnetfeld in der Mitte zweier Helmholtzspulen B ist dabei gegeben durch
\begin{equation}
    \text{B}= \text{$\mu$}_0 \frac{8}{\sqrt{125}} \frac{\text{NI}}{\text{R}}\\
\end{equation}
\noindent(N = Windungszalh, I = Spulenstrom, R = Spulenradius, $\mu_0$ = 4$\pi\cdot 10^{-7}$ Vs/Am).\cite{AL}\\
\noindent
Zu Beginn der Versuchsreihe muss die Achse der Kathodestrahlröhre mit Hilfe eines Deklinatorium-Iklinatorium parallel zum Horizontalkomponente des Erdmagnefeldes ausgerichtet werden.
Dieser Teil des Versuch wird bei der konstanter Beschleunigungspannung von $\text{U}_\text{B}$ = 250 und 420 V durchgefürht. 
Die Lage des Leuchfleckes bei B = 0 wird auf oberste des Koordinatennetzes auf dem Bilschirm gelegt.
\subsection{ Bestimmung der Intensität des lokalen Erdmagnetfeldes}
Die Lage des Leuchfleckes im xy-Koordinaten wird  schritteweise betrachtet und notiert bei möglichst niedriger Beschleunigungspannung ($\text{U}_\text{B}$ =200 V).
Zuerst wird die Achse der Röhre in Nord-Süd-Richtung und in Ost-West-Richtung gedreht. 
Die Elektronen werden durch Wirchkung des Erdfeldes in y-Richtung gelenkt.
 Das Helmholtz-Feld wird eingeschaltet und dessen Strom $\text{I}_{\text{hor}}$ wird solange verändert, bis ihr Magnetfeld das der Erde kompensiert hat.
Die Horizontalkomponente $\text{B}_{\text{hor}}$ des Erdfeldes ist daher gleich dem Helmholtz-Feld.
Die Totalintensität $\text{B}_{\text{total}}$ wird durch $\text{B}_{\text{hor}}$ und dem Inklinationswinkel $\varphi$ bestimmt.


\label{sec:Durchführung}

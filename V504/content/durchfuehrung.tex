\section{Durchführung}
\subsection{ Bestimmung der Kennlinie}

An die Hochvakuumdiode werden zwei Konstantspannungsgeräte angeschlossen. Mit einem
wird die Heizspannung angelegt, mit dem anderen wird die Anodenspannung gemessen.
Es werden drei verschiedene Heizströme angelegt und drei Kennlinien aufgenommen. Es
wird ein Spannungsbereich ausgewählt, in dem die Sättigung liegt.

\subsection{Bestimmung der Anlaufstromkurve}
Der Aufbau wird nun verändert. Mit der einen Spannungsquelle wird nun ein Gegenfeld
erzeugt. Diese Gegenspannung wird erhöht, die Heizspannung bleibt konstant maximal
eingestellt und der Anodenstrom wird mit einem nA-Meter gemessen.


\label{sec:Durchführung}

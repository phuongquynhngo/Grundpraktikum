\section{Anhang}
\begin{table}[H]
    \centering
    \caption{Messdaten für das Absorptionsspektrum von Brom.}
    \label{tab:Brom}
    \begin{tabular}{|c| c|}
    \toprule 
    $\theta/\mathrm{°}$ & $N/(\mathrm{Imp/s})$ \\
    \midrule
    12,8 &	10,0 \\
    12,9 &	12,0 \\
    13,0 &	9,0 \\
    13,1 &	13,0 \\
    13,2 &	18,0 \\
    13,3 &	21,0 \\
    13,4 &	25,0 \\
    13,5 &	27,0 \\
    13,6 &	27,0 \\
    13,7 &	22,0 \\
    13,8 &	25,0 \\
    13,9 &	21,0 \\
    14,0 &	23,0 \\
    14,1 &	20,0 \\
    14,2 &	21,0 \\
    14,3 &	19,0 \\
    \bottomrule
    \end{tabular}
\end{table}

\begin{figure}[H]
    \centering
    \includegraphics{Brom.pdf}
    \caption{Absorptionsspektrum von Brom.}
    \label{fig:Brom}
  \end{figure}

\begin{table}[H]
    \centering
    \caption{Messdaten für das Absorptionsspektrum von Gallium.}
    \label{tab:Gallium}
    \begin{tabular}{|c| c|}
    \toprule
    $\theta/\mathrm{°}$ & $N/(\mathrm{Imp/s})$ \\
    \midrule
    17,0 &	66,0 \\
    17,1 &	66,0 \\
    17,2 &	78,0 \\
    17,3 &	88,0 \\
    17,4 &	102,0 \\
    17,5 &	116,0 \\
    17,6 &	121,0 \\
    17,7 &	121,0 \\
    17,8 &	122,0 \\
    17,9 &	122,0 \\
    18,0 &	119,0 \\
    18,1 &	114,0 \\
    18,2 &	110,0 \\
    18,3 &	108,0 \\
    18,4 &	104,0 \\
    18,5 &	110,0 \\
    18,6 &	110,0 \\
    18,7 &	109,0 \\
    18,8 &	99,0 \\
    18,9 &	100,0 \\
    19,0 &	98,0 \\
    \bottomrule
    \end{tabular}
\end{table}
\begin{figure}[H]
    \centering
    \includegraphics{Gallium.pdf}
    \caption{Absorptionsspektrum von Gallium.}
    \label{fig:Gallium}
  \end{figure}

\begin{table}[H]
    \centering
    \caption{Messdaten für die Absorptionsspektrum von Rubidium.}
    \label{tab:Rubidium}
    \begin{tabular}{|c| c|}
    \toprule
    $\theta/\mathrm{°}$ & $N/(\mathrm{Imp/s})$ \\
    \midrule
    11,2 &  11,0 \\
    11,3 &	10,0 \\
    11,4 &	10,0 \\
    11,5 &	12,0 \\
    11,6 &	17,0 \\ 
    11,7 &	32,0 \\
    11,8 &	39,0 \\
    11,9 &	47,0 \\
    12,0 &	57,0 \\
    12,1 &	64,0 \\
    12,2 &	61,0 \\
    12,3 &	57,0 \\
    12,4 &	54,0 \\
    12,5 &	54,0 \\
    \bottomrule
    \end{tabular}
\end{table}
\begin{figure}[H]
    \centering
    \includegraphics{Rubidium.pdf}
    \caption{Absorptionsspektrum von Rubidium.}
    \label{fig:Rubidium}
  \end{figure}

\begin{table}[H]
    \centering
    \caption{Messdaten für die Absorptionsspektrum von Strontium.}
    \label{tab:Strontium}
    \begin{tabular}{|c| c|}
    \toprule
    $\theta/\mathrm{°}$ & $N/(\mathrm{Imp/s})$ \\
    \midrule
    10,5 &	43,0 \\
    10,6 &	41,0 \\
    10,7 &	40,0 \\
    10,8 &	44,0 \\
    10,9 &	50,0 \\
    11,0 &	89,0 \\
    11,1 &	120,0 \\
    11,2 &	152,0 \\
    11,3 &	181,0 \\
    11,4 & 	193,0 \\
    11,5 &	181,0 \\
    11,6 &	196,0 \\
    11,7 &	181,0 \\
    11,8 &	173,0 \\
    11,9 &	166,0 \\
    12,0 &	159,0 \\
    \bottomrule
    \end{tabular}
\end{table}
\begin{figure}[H]
    \centering
    \includegraphics{Strontium.pdf}
    \caption{Absorptionsspektrum von Strontium.}
    \label{fig:Strontium}
  \end{figure}

\begin{table}[H]
    \centering
    \caption{Messdaten für die Absorptionsspektrum von Zink.}
    \label{tab:Zink}
    \begin{tabular}{|c| c|}
    \toprule
    $\theta/\mathrm{°}$ & $N/(\mathrm{Imp/s})$ \\
    \midrule
    18,0 &	58,0 \\
    18,1 &	54,0 \\
    18,2 &	55,0 \\
    18,3 &	54,0 \\
    18,4 &	54,0 \\
    18,5 &	55,0 \\
    18,6 &	65,0 \\
    18,7 &	84,0 \\
    18,8 &	91,0 \\
    18,9 &	100,0 \\
    19,0 & 	102,0 \\
    19,1 &	100,0 \\
    19,2 &	98,0 \\
    19,3 &	100,0 \\
    19,4 &	95,0 \\
    19,5 &	98,0 \\
    \bottomrule
    \end{tabular}
\end{table}
\begin{figure}[H]
    \centering
    \includegraphics{Zink.pdf}
    \caption{Absorptionsspektrum von Zink.}
    \label{fig:Zink}
  \end{figure}

\begin{table}[H]
    \centering
    \caption{Messdaten für die Absorptionsspektrum von Zirkonium.}
    \label{tab:Zirkonium}
    \begin{tabular}{|c| c|}
    \toprule
    $\theta/\mathrm{°}$ & $N/(\mathrm{Imp/s})$ \\
    \midrule
    9,5	&   112,0 \\
    9,6	&   120,0 \\
    9,7	&   126,0 \\
    9,8	&   147,0 \\
    9,9	&   180,0 \\ 
    10,0 &  225,0 \\
    10,1 &	266,0 \\
    10,2 &	282,0 \\
    10,3 &	290,0 \\
    10,4 &	301,0 \\
    10,5 &	295,0 \\
    10,6 &	283,0 \\
    10,7 &	296,0 \\
    10,8 &	283,0 \\
    10,9 &	286,0 \\
    11,0 &	286,0 \\
    \bottomrule
    \end{tabular}
\end{table}
\label{sec:Auswertung}
\begin{figure}[H]
    \centering
    \includegraphics{Zirkonium.pdf}
    \caption{Absorptionsspektrum von Zirkonium.}
    \label{fig:Zirkonium}
  \end{figure}

\label{sec:Anhang}
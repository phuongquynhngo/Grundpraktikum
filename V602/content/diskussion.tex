\section{Diskussion}
\label{sec:Diskussion}
Die Abweichungen werden im folgenden mit der Formel
\begin{equation}
    \text{Abweichung $f$} = \Big\vert\frac{\text{Berechnete Werte} - \text{Literaturwerte}}{\text{Berechnete Werte}}\Big\vert \cdot 100 \%
\end{equation} berechnet.\\
Bei dem Versuch wurden folgende Einflüsse (systemmatische Fehler) nicht berücksichtigt:
\begin{enumerate}
    \item Ablesefehler bei Messgeräten
    \item Röntgenröhre falsch eingestellt
    \item Fehler in den Messgeräten bzw. Messmitteln
    \item Schwankungen der Netzspannung
    \item Ablesefehler von Grafiken
\end{enumerate}

\subsection{Überprüfung der Bragg-Bedingung}
Zur Überprüfung der Bragg-Bedingung beträgt der Winkel des Maximums den Wert von $\theta_\text{GM}=28,2\,\mathrm{°}$.
Es ergibt sich eine Abweichung $f_{\theta_\text{GM}}$ zum theoretischen Wert $\theta_\text{GM,theorie}=2\theta=28\,\mathrm{°}$ von 0,79\%. 
Somit ist die Bragg-Bedingung erfüllt.\\

\subsection{Analyse eines Emissionsspektrums der Kupfer-Röntgenröhre}
In Tabelle \ref{tab:Cuwerte} sind die Werte und deren mit der Gleichung (21) berechneten Abweichung $f$ zu den theoretischen Werte \cite{XT} von der Cu-$K_\alpha-$und Cu-$K_\beta-$Linie bei der Verwendung eines LiF-Kristalls \(d=204,1\, \mathrm{pm}\).
\begin{table}[H]
    \centering
    \caption{Werte für die Cu-$K_\alpha-$und Cu-$K_\beta-$Linie.}
    \label{tab:Cuwerte}
    \begin{tabular}{|l  ||c|c|c ||c|c|c| }
        \toprule
        {} &  $E{_K ^\text{Lit}}/\mathrm{keV} $ & $E{_K}/\mathrm{keV} $   & $f_{E_K}/\mathrm{\%}$ &     $\theta^\text{Lit}/\mathrm{°}$    &    $\theta/\mathrm{°}$  &  $f_{\theta}/\mathrm{\%}$ \\
        \midrule
        $K_{\alpha }$ &8,048&8,059&0,316&22,53&22,5&0,133\\
        $K_{\beta}$&8,905&8,931&0,291&20,26&20,2&0,297\\
        \bottomrule
        \end{tabular}
\end{table}
Die Werte der Halbwertsbreite (Full Width at Half Maximum) für die Cu-$K_\alpha-$ und Cu-$K_\beta-$Linie sind:\\

$K_\alpha$-Linie: $\text{FWHM}_\alpha= 0,494 \,\mathrm{°}$, \(\Delta E_{\text{FWHM},\alpha}=166,231\,\mathrm{eV}\)\\

$K_\beta$-Linie: $\text{FWHM}_\beta= 0,495 \,\mathrm{°}$, \(\Delta E_{\text{FWHM},\beta}=207,488\,\mathrm{eV}\)\\

Das Auflösungsvermögen der Apparatur beider Linien lauten
\begin{align*}
    A_\alpha &= 48,481\\
    A_\beta &= 43,043.\\
  \end{align*}

Die Energiedifferenz zwischen die $K_\alpha$- und $K_\beta$-Linie beträgt 
  \begin{align*}
      \Delta E  = 872\, \mathrm{eV}.
    \end{align*}
Die Abschirmkonstanten $\sigma_1$, $\sigma_2$, $\sigma_3$ für Kupfer und deren Abweichungen mit den theoretischen Werten sind in Tabelle \ref{tab:Kupferab} gegeben.

\begin{table}[H]
    \centering
    \caption{Abschirmkonstanten für Kupfer.}
    \label{tab:Kupferab}
    \begin{tabular}{|l  ||c|c|c| }
        \hline
        {}&$\sigma_{1,\text{theorie}}$&{}&{}\\
        \hline
        $\text{Für}\,\, E{^\text{Lit}_{K,\text{abs}}}$&3,299&{}&{} \\
        \hline
        \hline
        {}&$\sigma_{2,{\text{theorie}}}$&$\sigma_2$&$f_{\sigma_2}/\mathrm{\%}$\\
        \hline
        $\text{Für}\,\, E_{K,\alpha }$ &12,382&12,480&0,785\\
        \hline
        \hline
        {}&$\sigma_{3,{\text{theorie}}}$&$\sigma_3$&$f_{\sigma_3}/\mathrm{\%}$\\
        \hline
        $\text{Für}\,\, E_{K,\beta }$&21,621&22,896&5,569\\
        \hline
        \end{tabular}
\end{table}
\subsection{Anaylse der Absorptionsspektren}
Tabelle \ref{tab:Absorber} zeigt die ermittelten und die Literaturwerte \cite{XT} der unterschiedlichen Absorptionspektren. 
Tabelle \ref{tab:Abweichungabsorber} zeigt die relativen Abweichungen, die sich mit der Gleichung (21) berechnet werden.
\begin{table}[H]
    \centering
    \caption{Werte für die Absorber.}
    \label{tab:Absorber}
    \begin{tabular}{|l ||c|c || c|c || c|c| }
        \toprule
        {}  &$E{^{\text{Lit}}_K}/\mathrm{keV} $ & $E{_K}/\mathrm{keV} $   & $\theta{^{\text{Lit}}_K}/\mathrm{°}$  & $\theta{_K}/\mathrm{°}$  & $\sigma{^{\text{Lit}}_K}$ & $\sigma{_K}$ \\
        \midrule
        $\text{Zn}$&9,669&9,619&18,6&18,7&3,545&3,615\\
        $\text{Ga}$&10,378&10,284&17,29&17,45&3,605&3,731\\
        $\text{Br}$&13,484&13,455&13,22&13,25&3,838&3,872\\
        $\text{Rb}$&15,208&15,144&11,70&11,75&3,942&4,014\\
        $\text{Sr}$&16,115&15,947&11,03&11,15&3,990&4,172\\
        $\text{Zr}$&18,008&17,848&9,86&9,95&4,091&4,255\\
        \bottomrule
        \end{tabular}
\end{table}

\begin{table}[H]
    \centering
    \caption{Relative Abweichung.}
    \label{tab:Abweichungabsorber}
    \begin{tabular}{|l|c|c|c|c|}
        \toprule
        {} &$f_{E_K}/\mathrm{\%}$ &$f_{\theta_K}/\mathrm{\%}$ &$f_{\sigma_K}/\mathrm{\%}$  \\
        \midrule
        $\text{Zn}$&0,520&0,535&1,936\\
        $\text{Ga}$&0,914&0,917&3,377\\
        $\text{Br}$&0,216&0,226&0,878\\
        $\text{Rb}$&0,423&0,426&1,769\\
        $\text{Sr}$&1,053&0,011&4,362\\
        $\text{Zr}$&0,896&0,905&3,854\\

        \bottomrule
        \end{tabular}
\end{table}


Der experimentelle Wert der Rydbergfrequenz beträgt \(R =(3,0598 \pm 0,0349) \cdot 10^{15}\,\mathrm{Hz}\).
Der Theoriewert  von der Rydbergfrequenz lautet \(R =3,2898 \cdot 10^{15}\,\mathrm{Hz}\) \cite{RB}.
Für die relative Abweichung ergibt sich 7,517\%. 

Der experimentelle Wert der Rydbergenergie beträgt \(R_{\infty} =(12,671 \pm 0,1445)\,\mathrm{eV}\).
Der Theoriewert  von der Rydbergenergie lautet \(R_{\infty} =13,606\,\mathrm{eV}\) \cite{RB}.
Für die relative Abweichung ergibt sich 7,376\%. 

Die Abweichungen betragen gro{\ss}e Werte, weil es nur sechs Werte für die Ausgleichsrechnung zur Bestimmung der Rydbergkonstante genutzt wurden.

\section{Diskussion}
\label{sec:Diskussion}
Die Abweichung wird im folgenden mit der Formel
\begin{equation}
    \text{Abweichung} \,f = \Big\vert\frac{\text{Berechnete Werte}-\text{Literaturwerte}}{\text{Literaturwerte}}\Big\vert \cdot 100 \%
\end{equation}
berechnet. 
Bei dem Versuch werden folgende Einflüsse (systemmatische Fehler) nicht berücksichtigt:
\begin{enumerate}
    \item Ablesefehler bei Messgeräten
    \item Fehler in den Messgeräten bzw. Messmitteln
    \item Schwankungen der Netzspannung
\end{enumerate}
Der experimentelle Wert der Halbwertzeit der Vanadiumprobe beträgt \(T=(186,130 \pm 12,295)\,\mathrm{s}\).
Der Theoriewert  von der Halbwertzeit der Vanadiumprobe lautet \(T = 224,6\,\mathrm{s}\) \cite{2}.
Für die relative Abweichung ergibt sich 17,128\%. \\
Der experimentelle Wert der Halbwertzeit der Rhodiumprobe beträgt \(T = (213,341 \pm 28,235)\,\mathrm{s}\).
Der Theoriewert  von der Halbwertzeit der Rhodiumprobe lautet \(T = 260\,\mathrm{s}\)\cite{3}.
Für die relative Abweichung ergibt sich 17,946\%.

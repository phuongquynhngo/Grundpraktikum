\section{Diskussion}
Bei diesem Versuch wird die $^{204} Tl-\text{Quelle}$ so plaziert, dass bei einer mittleren Zahlrohrspannung eine Zahlrate von 100 Imp/s nicht überschritten wurde.
Es ist wichtig zur Vermeidung von Totzeit-Korrekturen und von Verkürzung der Lebensdauer des Zahlrohres, weil die Lebensdauer eines mit organischen Substanzen gefüllten Zählrohres ist bei so grö\ss er Schwingungsamplitude begrenzt ist .

\subsection{Aufnahme der Geiger-Müller Charakteristik}
Die  Integrationszeit  pro  Zahlrohrspannung  in dem Versuchteil \( \Delta t=60\,\text{s}\) wird so gewählt, damit die Zahlrate im Geiger-Plateau in der Großenordnung von \(N=10000\, \mathrm{Imp}\) liegt. 
Da der Plateauanstieg sehr gering ist, müssen dabei so viele Werte wie möglich sehr genau ausgemessen werden. Au\ss erdem wird die solche Messzeit gewählt, dass der relative statistische Fehler jedes Messpunktes bei <1\% liegt.\\
Bei einem idealen Zählrohr soll der Plateauanstieg null betragen.
Experimentiell besitzt der Anstieg den Wert i.h.v 2,104\%. Die Abweichung zum Vergleich mit idealem Wert ist ziemlich gering. 
Eine Fehlerquelle stellt dar, dass der Bereich des Geiger-Plateaus nicht genau bestimmt und nach Einschätzung gewählt wird.

\subsection{Bestimmung der Totzeit $T$}
Bei der Bestimmung der Totzeit wird keine Aussage über die Richtigkeit und Genauigkeit der Totzeit getroffen.\\ 
Durch die Zwei-Quellen-Methode ist die Totzeit \( T =(114,957 \pm 47,273)\,\mu \text{s}\) bestimmt mit der erfüllten Bedingung \({N}_{1+2} < {N}_1 + {N}_2\) \( (158479 \,\text{Imp} < 172559\,\text{Imp}) \).\\
Mit dem Oszilloskop ist die Totzeit $T=100\,\mu \text{s}$ ungenau abzulesen, weil die Impulse in dem Bild von Momentaufnahme sehr schwer zu erkennen ist. 
Aus diesem Grund ist die Erholungszeit durch Oszilloskop auch nicht zu bestimmen.

\subsection{Bestimmung des Zählrohrstroms}
Bei der Bestimmung des Zählrohrstroms wird der Strom abgelesen. Der Ableserfehler ist ein Grund für die gro\ss en Fehler von der Zahl der freigesetzten Ladungen pro eingefallenen Teilchen $Z$ so wie von Zählrohrstrom.
Trotz der Abweichung weisen die Daten insgesamt einen linearen Verlauf auf (lineare Regression).

\label{sec:Diskussion}
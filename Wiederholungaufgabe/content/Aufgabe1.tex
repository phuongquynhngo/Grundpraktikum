\section{Aufgabe 1}
\label{sec:Aufgabe 1}
Fassen Sie die Bedeutung folgender Begriffe in Ihre Worte und notieren Sie eine Möglichkeit der Berechnung.


1.1 Was bezeichnet der Mittelwert?

 Mittelwert wird berechnet, indem die alle Werte addiert werden und diese Summe teilt durch ihre Anzahl der Werten.

 Die Formel für den Mittelwert lautet
 \begin{equation}
    \bar{x}=\frac{1}{n}\sum_{i=1}^n x_i .
\end{equation}



1.2 Welche Bedeutung hat die Standardabweichung?

Bedeutung der Standardabweichung des Mittelwertes: Die durchschnitte Abweichung der allen Werten von Mittelwert.

Die Formel für die Standardabweichung lautet
\begin{equation}
    s_x=\sqrt{\frac{1}{n-1}\sum_{i=1}^n {(x_i-\bar{x})^2 }}.
\end{equation}



1.3 Worin unterschied sich die Streuung der Messwerte und der Fehler des Mittelwertes?

Streuung der Messwerte ( auch Standardabweichung) beschreibt, wie die einzelnen Werte um den Mittelwert allen Werte verteilen bzw. liegen.

Fehler des Mittelwertes beschreibt, wie die Mittelwerte der Stichproben um den tatsächliche Mittelwert verteilen.

Die Formel für den Fehler des Mittelwertes lautet
\begin{equation}
    \Delta \bar{x} = s_{\bar{x}}=\frac{s_x}{\sqrt{n}}= \sqrt{\frac{1}{n(n-1)}\sum_{i=1}^n {(x_i-\bar{x})^2 }}.
\end{equation}




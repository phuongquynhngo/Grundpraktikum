\section{Aufgabe 2}
\label{sec:Aufgabe 2}
Berechnen Sie das Volumen eines Hohlzylinders, mit $R_{\text{innen}}=(10 \pm 1)$ cm, $R_{\text{au"sen}}=(15 \pm 1)$ cm und $h=(20 \pm 1)$ cm
       
Die Formel für das Volumen eines Hohlzylinders lautet
\begin{equation}
    V=\pi \cdot h \cdot({R_{\text{au"sen}}}^2-{R_{\text{innen}}}^2).
\end{equation}

Der Fehler für $V$ wird dabei über die Gau"s´sche Fehlerfortfplanzung 
\begin{equation}
    \Delta f= \sqrt{\sum_{i=1}^N {\Bigl(\frac{\partial f}{\partial y_i}\Bigr)^2 (\Delta y_i)^2}}
\end{equation}
berechnet: 
\begin{align*}
    \Delta V= \sqrt{(\pi \cdot \Delta h \cdot({R_{\text{au"sen}}}^2-{R_{\text{innen}}}^2))^2 + (2 \pi \cdot h \cdot R_{\text{au"sen}} \cdot \Delta R_{\text{au"sen}} )^2 + (2 \pi \cdot h \cdot R_{\text{innen}} \cdot \Delta R_{\text{innen}} )^2 }
\end{align*}

Somit ergibt sich als Volumen des Hohlzylinders 
\begin{align*}
    V &= (7853,98 \pm 105,37) \mathrm{cm^3}. \\
\end{align*}
